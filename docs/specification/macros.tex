\usepackage{jhupllab}
\usepackage{amsmath}
\usepackage{amssymb}
%\usepackage[T1]{fontenc}
\usepackage{mathpartir}
\usepackage{accents}
\usepackage{wasysym}
\usepackage{stmaryrd}
\usepackage{mathtools}
\usepackage{amsthm}
\usepackage{thmtools}
\usepackage{ifthenx}
\usepackage{enumitem}
\usepackage{tabto}

\usepackage{tikz}
\usetikzlibrary{arrows}
\usetikzlibrary{calc}
\usetikzlibrary{decorations.markings}
\usetikzlibrary{shapes}
\usetikzlibrary{positioning}

% Create a background layer
\pgfdeclarelayer{background}
\pgfsetlayers{background,main}

\let\oldrmdefault\rmdefault
\let\oldsfdefault\sfdefault
\usepackage{lmodern}
\let\rmdefault\oldrmdefault
\let\sfdefault\oldsfdefault

\newcommand{\bbrule}[4][]{\inferrule*[lab={\sc #2},#1]{#3}{#4}}
% Definition to use \arcr in lieu of \\ for arrays inside of mathpars
\makeatletter
\def\arcr{\@arraycr}
\makeatother

\declaretheorem[within=section]{lemma}
\declaretheorem[name=Definition,numberlike=lemma]{definition}
\declaretheorem[name=Notation,numberlike=lemma]{notation}
\declaretheorem[name=Theorem,numberlike=lemma]{theorem}

\numberwithin{figure}{section}

\newcommand{\fnstyle}[1]{\text{\textsmaller{\sc{#1}}}}
\newcommand{\deffn}[2]{%
    \expandafter\newcommand\expandafter{\csname #1\endcsname}[1][]{\fnstyle{#2}\ifthenelse{\isempty{##1}}{}{(##1)}}%
}

\newcommand{\formalRuleLine}{\\ \hspace*{1.5em}}

\newcommand{\ggc}{g}

\renewcommand{\setof}[2][]{{\color{red}\textrm{\bfseries We are not using \texttt{\char`\\setof} in this document!}}}

\newsavebox{\codeBoxA}
\newsavebox{\codeBoxB}

%%%%%%%%%%%%%%%%%%%%%%%%%%%%%%%%%%%%%%%%%%%%%%%%%%%%%%%%%%%%%%%%%%%%%%%%%%%%%%
% Expression grammar

\newcommand{\tttilde}{{\raisebox{-0.25em}{\texttt{\char`\~}}}}
\newcommand{\codetilde}{\tttilde}

% edited these to get rid of italics in definitions etc.  surely not optimal defns but working - SS
\defgt[gtcolon]{\texttt{\upshape :}}
\defgt[gttilde]{\texttt{\upshape \codetilde}}
\defgt[gtquestion]{\texttt{\upshape ?}}
\defgt[gtarrow]{\texttt{\upshape ->}}
\newcommand{\gtdot}{\texttt{.}}
\defgt[gtob]{\upshape \char`\{}
\defgt[gtcb]{\upshape \char`\}}
\defgt[gtop]{\upshape (}
\defgt[gtcp]{\upshape )}
\defgt[gtis]{\texttt{\upshape =}}
\defgt[gtfun]{\upshape fun}
\defgt[gtref]{\upshape ref}
\defgt[gtset]{\upshape <-}
\defgt[gtlbrack]{\upshape \lbrack}
\defgt[gtrbrack]{\upshape \rbrack}
\defgt[gtdoublequote]{\upshape "}
\defgt[gttrue]{\upshape true}
\defgt[gtfalse]{\upshape false}
\defgt[gtstring]{\upshape string}
\defgt[gtany]{\upshape any}

\defgn[binop]{\odot}
\defgn[unop]{\Box}

\defgt[gtplus]{\texttt{\upshape +}}
\defgt[gtminus]{\texttt{\upshape -}}
\defgt[gtless]{\texttt{\upshape <}}
\defgt[gtand]{\texttt{\upshape and}}
\defgt[gtor]{\texttt{\upshape or}}
\defgt[gtleq]{\texttt{\upshape <=}}
\defgt[gteq]{\texttt{\upshape ==}}
\defgt[gtat]{\texttt{\upshape @}}

\defgn[expr]{e}
\defgn[env]{E}
\defgn[ecl]{c}
\defgn[elb]{b}
\defgn[eval]{v}
\defgn[elbl]{\ell}
\defgn[efunc]{f}
\defgn[erec]{r}
\defgn[epat]{p}
\defgn[ev]{x}
\defgn[eint]{\mathbb{Z}}
\defgn[ebool]{\mathbb{B}}
\defgn[estring]{\mathbb{S}}

\newcommand{\conditional}[4]{#1 \gttilde #2 \gtquestion #3 \gtcolon #4}

%%%%%%%%%%%%%%%%%%%%%%%%%%%%%%%%%%%%%%%%%%%%%%%%%%%%%%%%%%%%%%%%%%%%%%%%%%%%%%
% Operational semantics

\newcommand{\smallstep}{\nobreak\longrightarrow^1\nobreak}
\newcommand{\smallsteps}{\nobreak\longrightarrow^*\nobreak}
\newcommand{\nsmallstep}{\nobreak\longarrownot\longrightarrow^1}

\newcommand{\efreshen}[2]{\mbox{\textsc{fr}}(#1,#2)}

\deffn{rv}{RV}
\newcommand{\wire}[3]{\text{\textsmaller{\textsc{Wire}}}\ifthenelse{\isempty{#1}}{}{(#1,#2,#3)}}

\newcommand{\fnIntOp}[3]{\text{\textsmaller{\textsc{IntOp}}}\ifthenelse{\isempty{#1}}{}{(#1,#2,#3)}}
\newcommand{\fnBoolBinOp}[3]{\text{\textsmaller{\textsc{BoolBinOp}}}\ifthenelse{\isempty{#1}}{}{(#1,#2,#3)}}
\newcommand{\fnBoolUnOp}[2]{\text{\textsmaller{\textsc{BoolUnOp}}}\ifthenelse{\isempty{#1}}{}{(#1,#2)}}
\newcommand{\fnStringBinOp}[3]{\text{\textsmaller{\textsc{StringBinOp}}}\ifthenelse{\isempty{#1}}{}{(#1,#2,#3)}}

%%%%%%%%%%%%%%%%%%%%%%%%%%%%%%%%%%%%%%%%%%%%%%%%%%%%%%%%%%%%%%%%%%%%%%%%%%%%%%
% Graph-based operational semanitcs

\defgn[gdep]{d}
\defgn[gdeps]{D}
\defgn[gacl]{a}

\newcommand{\annotated}[4]{#3 \stackrel{#1 #2}{\gtis} #4}

\defgn[gacls]{\toDeprecate{A}}
\defgn[evals]{V}
\defgn[gxstack]{X}
\defgn[gcstack]{C}

\newcommand{\gstart}[1]{\textrm{\textsmaller{\sc Start}}\ifthenelse{\isempty{#1}}{}{\ensuremath{(#1)}}}
\newcommand{\gend}[1]{\textrm{\textsmaller{\sc End}}\ifthenelse{\isempty{#1}}{}{\ensuremath{(#1)}}}

\newcommand{\before}{\mathrel{\mathrlap{<}{\ <}}}
\newcommand{\isbefore}{\ensuremath{\mathrel{\mathrlap{<}{\ \mathnormal{\lhd}}}}}

\newcommand{\gmodeEnter}{\mathnormal{\Leftcircle\!\!}}
\newcommand{\gmodeExit}{\mathnormal{\!\!\Rightcircle}}

\deffn{embed}{Embed}

\deffn{succs}{Succs}
\deffn{preds}{Preds}
\newcommand{\activenode}[2]{\fnstyle{Active}\ifthenelse{\isempty{#1}}{}{(#1\ifthenelse{\isempty{#2}}{}{,#2})}}

\newcommand{\glookupOld}[4][\gdeps]{\toDeprecate{#1(#2,#3\ifthenelse{\isempty{#4}}{}{,#4})}}
\newcommand{\glookup}[5][\gdeps]{#1(#2,#3\ifthenelse{\isempty{#4}}{}{,#4,#5})}

\newcommand{\gwire}[4]{\text{\textsmaller{\textsc{Wire}}}(#1,#2,#3,#4)}

%%%%%%%%%%%%%%%%%%%%%%%%%%%%%%%%%%%%%%%%%%%%%%%%%%%%%%%%%%%%%%%%%%%%%%%%%%%%%%
% Type system grammar

\newcommand{\typeannot}{\hat}

\defgn[texpr]{\typeannot e}
\defgn[tcl]{\typeannot c}
\defgn[tlb]{\typeannot b}
\defgn[tval]{\typeannot v}
\defgn[tlbl]{\ell}
\defgn[tfunc]{\typeannot f}
\defgn[trec]{\typeannot r}
\defgn[tpat]{\typeannot p}
\defgn[tv]{\typeannot x}
\defgt[gtint]{\upshape int}

\defgn[tgdep]{\typeannot d}
\defgn[tgdeps]{\typeannot D}
\defgn[tgacl]{\typeannot a}

\defgn[tgacls]{\typeannot A} % used in proof discussion
\defgn[tvals]{\typeannot V}
\defgn[tgxstack]{\typeannot X}
\defgn[tgcstack]{\typeannot C}
\defgn[tgcstacks]{\boldsymbol{\typeannot C}}

\newcommand{\tgcstackmodel}{\Sigma}
\newcommand{\tgcemptystack}{\epsilon}

\deffn{stpop}{Pop}
\makeatletter
\newcommand{\stpush@}[1][]{
    \ifthenelse{\isempty{#1}}{}{,#1})
}
\newcommand{\stpush}[1][]{
    \fnstyle{Push}\ifthenelse{\isempty{#1}}{}{(#1\stpush@}
}
\newcommand{\stistop@}[1][]{
    \ifthenelse{\isempty{#1}}{}{,#1})
}
\newcommand{\stistop}[1][]{
    \fnstyle{IsTop}\ifthenelse{\isempty{#1}}{}{(#1\stistop@}
}
\makeatother
\deffn{stisprecise}{\toDeprecate{IsPrecise}}
\deffn{stscratch}{\toDeprecate{Scratch}}
\deffn{stendofblock}{EndOfBlock}

\newcommand{\tglookup}[5][\tgdeps]{#1(#2,#3\ifthenelse{\isempty{#4}}{}{,#4,#5})}

%%%%%%%%%%%%%%%%%%%%%%%%%%%%%%%%%%%%%%%%%%%%%%%%%%%%%%%%%%%%%%%%%%%%%%%%%%%%%%
% Soundness notation

\newcommand{\bsim}{\cong}

\newcommand{\graphsimby}{\preccurlyeq}

%%%%%%%%%%%%%%%%%%%%%%%%%%%%%%%%%%%%%%%%%%%%%%%%%%%%%%%%%%%%%%%%%%%%%%%%%%%%%%
% Extensions

\defgt[gtderef]{\upshape !}
\defgt[gtnot]{\upshape not}
\newcommand{\matches}[3]{{\text{\textsmaller{\sc Matches}}}\ifthenelse{\isempty{#1}}{}{(#1,#2,#3)}}
\newcommand{\unique}[1]{{\text{\textsmaller{\sc Unique}}}\ifthenelse{\isempty{#1}}{}{(#1)}}
\newcommand{\bindings}[1]{{\text{\textsmaller{\sc Bindings}}}\ifthenelse{\isempty{#1}}{}{(#1)}}
\newcommand{\bindingsp}[1]{{\text{\textsmaller{\sc Bindings}}'}\ifthenelse{\isempty{#1}}{}{(#1)}}
\newcommand{\defines}[1]{{\text{\textsmaller{\sc Defines}}}\ifthenelse{\isempty{#1}}{}{(#1)}}
\newcommand{\uses}[1]{{\text{\textsmaller{\sc Uses}}}\ifthenelse{\isempty{#1}}{}{(#1)}}
\newcommand{\usesp}[1]{{\text{\textsmaller{\sc Uses}}'}\ifthenelse{\isempty{#1}}{}{(#1)}}
\newcommand{\immediatelyrequires}[3]{\ifthenelse{\isempty{#1}}{\prec}{#2 \prec_{#1} #3}}
\newcommand{\requires}[3]{\ifthenelse{\isempty{#1}}{\prec\!\!\!\prec}{#2 \prec\!\!\!\prec_{#1} #3}}
\newcommand{\availd}[2]{{\text{\textsmaller{\sc Avail}}^d}\ifthenelse{\isempty{#1}}{}{\!\!_{#1}(#2)}}
\newcommand{\availr}[2]{{\text{\textsmaller{\sc Avail}}^r}\ifthenelse{\isempty{#1}}{}{\!\!_{#1}(#2)}}
\newcommand{\inscope}[3]{{\text{\textsmaller{\sc InScope}}}\ifthenelse{\isempty{#1}}{}{_{#1}(#2,#3)}}
\newcommand{\wellformed}[1]{{\text{\textsmaller{\sc WellFormed}}}\ifthenelse{\isempty{#1}}{}{(#1)}}
\newcommand{\flatten}[1]{{\text{\textsmaller{\sc Ft}}}\ifthenelse{\isempty{#1}}{}{(#1)}}
\newcommand{\flatteni}[1]{{\text{\textsmaller{\sc Ft}}^i}\ifthenelse{\isempty{#1}}{}{(#1)}}
\newcommand{\deps}[2]{{\text{\textsmaller{\sc Deps}}}\ifthenelse{\isempty{#1}}{}{_{#1}(#2)}}
\newcommand{\uptofirstdependent}[2]{{\text{\textsmaller{\sc UFD}}}\ifthenelse{\isempty{#1}}{}{_{#1}(#2)}}

\newcommand{\negativePatternSelectCase}[3]{#1 \Vdash #2 \rightslice #3}

\newcommand{\tglookupSugar}[3][\tgdeps]{#1(#2,#3)}
\newcommand{\tglookupPSM}[4][\tgdeps]{#1(#2\ifthenelse{\isempty{#3}}{}{,#3,#4})}
\newcommand{\tState}[4]{\langle #1, #2, #3, #4 \rangle}
\defgn[tstates]{\typeannot S}
\defgn[tpats]{\Pi}
\defgn[tpatsp]{\Pi^+}
\defgn[tpatsn]{\Pi^-}
\defgn[tgkcapturesize]{\kappa}
\defgn[tgk]{\typeannot k}
\defgn[tfval]{\typeannot \varphi}
\defgn[tfvals]{\typeannot \Phi}
\defgn[tgkstack]{\typeannot K}

\newcommand{\filtered}[3]{#1^{#2}_{#3}}

\newcommand{\kvar}[3]{{#1{}^{#2}_{#3}}}
\newcommand{\kproject}[3]{{.#1{}^{#2}_{#3}}}
\newcommand{\kjump}[2]{{\text{\textsmaller{\sc Jump}}}(#1,#2)}
\newcommand{\kderef}[2]{{{\gtderef}{}^{#1}_{#2}}}
\newcommand{\kcapture}[1]{\text{\textsmaller{\sc Capture}}^{#1}}
\newcommand{\kvalueOld}[3]{\toDeprecate{{#1{}^{#2}_{#3}}}}
\newcommand{\krealflowhuh}{\text{\textsmaller{\sc RealFlow?}}}
\newcommand{\kaliashuh}{\text{\textsmaller{\sc Alias?}}}
\newcommand{\krewind}{\text{\textsmaller{\sc Rewind}}}
\newcommand{\kunop}{\text{\textsmaller{\sc UnOp}}}
\newcommand{\kbinop}{\text{\textsmaller{\sc BinOp}}}
\newcommand{\ksideeffectstart}{\text{\textsmaller{\sc SEStart}}}
\newcommand{\ksideeffectvar}[5]{{#1{}^{#2}_{#3}}@(#4,#5)}
\newcommand{\ksideeffectescape}[1]{\text{\textsmaller{\sc SEEscape}}(#1)}

\newcommand{\mayaliasAbs}[4]{\text{\textsc{\textsmaller{may-alias}}}\ifthenelse{\isempty{#1}}{}{(#1,#2,#3,#4)}}
\newcommand{\mustaliasAbs}[4]{\text{\textsc{\textsmaller{must-alias}}}\ifthenelse{\isempty{#1}}{}{(#1,#2,#3,#4)}}

\newcommand{\immediatelyMatchedBy}[1]{{%
    \text{\textsc{\textsmaller{ImmediatelyMatchedBy}}}\ifthenelse{\isempty{#1}}{}{(#1)}%
}}
\newcommand{\abstractBinaryOperation}[3]{{%
    \text{\textsmaller{\sc AbstractBinaryOperation}}\ifthenelse{\isempty{#1}}{}{(#1,#2,#3)}%
}}
\newcommand{\abstractUnaryOperation}[2]{{%
    \text{\textsmaller{\sc AbstractUnaryOperation}}\ifthenelse{\isempty{#1}}{}{(#1,#2)}%
}}

%%%%%%%%%%%%%%%%%%%%%%%%%%%%%%%%%%%%%%%%%%%%%%%%%%%%%%%%%%%%%%%%%%%%%%%%%%%%%%
% Rule macros

\newcommand{\rulename}[1]{%
    \begingroup%
        \setlength{\fboxsep}{1.5pt}%
        \fcolorbox{black}{gray!15!white}{%
            \textsc{\textsmaller{#1}}%
        }%
        \vspace*{1pt}%
    \endgroup%
    \\%
}

%%%%%%%%%%%%%%%%%%%%%%%%%%%%%%%%%%%%%%%%%%%%%%%%%%%%%%%%%%%%%%%%%%%%%%%%%%%%%%
% Listings code

\usepackage{listings}

\newcommand{\plangbasicstyle}{\ttfamily}
\newcommand{\plangkeywordstyle}{\plangbasicstyle \color{NavyBlue}}
\newcommand{\plangsymbolstyle}{\plangbasicstyle \color{RoyalPurple}}
\newcommand{\plangcommentstyle}{\plangbasicstyle \color{ForestGreen}}

\newcommand{\ttob}{\text{\upshape\ttfamily\char`\{}}
\newcommand{\ttcb}{\text{\upshape\ttfamily\char`\}}}
\newcommand{\ttop}{\text{\upshape\ttfamily(}}
\newcommand{\ttcp}{\text{\upshape\ttfamily)}}

\catcode`\#=11%
\newcommand{\es}{} % consumes whitespace
\newcommand{\plangset}{%
    \lstset{%
        basicstyle=\plangbasicstyle,
        fontadjust=false,
        showspaces=false,
        showtabs=false,
        numberstyle=\tiny\color{gray},
        stepnumber=1,
        numbers=left,
        numbersep=5pt,
        escapeinside={$}{$},
        escapebegin=$,
        escapeend=$,
        keywordstyle=\plangkeywordstyle,
        keywords={fun},
        commentstyle=\plangcommentstyle,
        morecomment=[l]{#},
        literate=%
            {\{}{\begingroup    \plangsymbolstyle   \ttob       \endgroup}{1}%
            {\}}{\begingroup    \plangsymbolstyle   \ttcb       \endgroup}{1}%
            {(}{\begingroup     \plangsymbolstyle   \ttop       \endgroup}{1}%
            {)}{\begingroup     \plangsymbolstyle   \ttcp       \endgroup}{1}%
            {=}{\begingroup     \plangsymbolstyle   =\es        \endgroup}{1}%
            {?}{\begingroup     \plangsymbolstyle   ?\es        \endgroup}{1}%
            {:}{\begingroup     \plangsymbolstyle   :\es        \endgroup}{1}%
            {~}{\begingroup     \plangsymbolstyle   \tttilde    \endgroup}{1}%
            {->}{\begingroup    \plangsymbolstyle   ->\es       \endgroup}{2}%
            {@\{}{\begingroup   \plangcommentstyle  \ttob       \endgroup}{1}%
            {@\}}{\begingroup   \plangcommentstyle  \ttcb       \endgroup}{1}%
            {@:}{\begingroup    \plangcommentstyle  :\es        \endgroup}{1}%
            {@=}{\begingroup    \plangcommentstyle  =\es        \endgroup}{1}%
    }%
}
\catcode`\#=6%

\plangset % just run it once to make things consistent
\lstnewenvironment{plang}[1][]{\begingroup\renewcommand{\plangbasicstyle}{\small\ttfamily} #1}{\endgroup}
\newcommand{\plangil}[1][]{\plangset{}#1\lstinline[mathescape]}

\newcommand{\camlset}{\lstset{language=caml,basicstyle=\ttfamily,escapeinside={$}{$},escapebegin=$,escapeend=$,morekeywords=case}}
\lstnewenvironment{caml}{\begingroup \camlset}{\endgroup}
\newcommand{\camlil}[1][]{\camlset{}#1\lstinline[mathescape]}

% Brace yourself.  The following terrible, glorious hack will squeeze the characters in listings closer together, since the
% existing spacing is more than a little too wasteful.
\makeatletter
\lst@Key{basewidth}{0.5em,0.35em}{\lstKV@CSTwoArg{#1}% was 0.6 and 0.45em resp - SS
    {\def\lst@widthfixed{##1}\def\lst@widthflexible{##2}%
     \ifx\lst@widthflexible\@empty
         \let\lst@widthflexible\lst@widthfixed
     \fi
     \def\lst@temp{\PackageError{Listings}%
                                {Negative value(s) treated as zero}%
                                \@ehc}%
     \let\lst@error\@empty
     \ifdim \lst@widthfixed<\z@
         \let\lst@error\lst@temp \let\lst@widthfixed\z@
     \fi
     \ifdim \lst@widthflexible<\z@
         \let\lst@error\lst@temp \let\lst@widthflexible\z@
     \fi
     \lst@error}}
\makeatother

%%%%%%%%%%%%%%%%%%%%%%%%%%%%%%%%%%%%%%%%%%%%%%%%%%%%%%%%%%%%%%%%%%%%%%%%%%%%%%
% TikZ Diagrams

\tikzstyle{clause}=[draw,ellipse,minimum width=4mm,minimum height=4mm,font=\small\ttfamily,inner sep=0.5mm,fill=green!20]
\tikzstyle{applclause}=[clause,thick,fill=gray!30]
\tikzstyle{wireclause}=[draw,rectangle,rounded corners,minimum width=4mm,minimum height=4mm,font=\small\ttfamily,inner sep=0.5mm,fill=orange!30]
\tikzstyle{flow}=[->]
\tikzstyle{wireflow}=[flow,densely dotted]
\tikzstyle{skippedflow}=[flow,dashed,gray!50]

% An example: the command \flowpath{a}{b}{c}{} generates the output
%   \draw[flow] (a) -- (b); \draw[flow] (b) -- (c);
% Note that the last empty group is necessary to stop the chain.
\newcommand{\flowpath}[2]{
    \ifthenelse{\isempty{#2}}{}{%
        \draw[flow] (#1) -- (#2);%
        \flowpath{#2}%
    }%
}

% Macros for diagrams
\newcommand{\subvar}[2]{%
    {\ttfamily #1\raisebox{-0.2em}{\hbox{\tiny\ttfamily #2}}}%
}

\newcommand{\calleq}[2]{\raisebox{-0.15em}{\hbox{\ensuremath{\stackrel{\scalebox{0.85}{\ensuremath{\text{\texttt{\tiny #1}}\scriptscriptstyle #2}}}{\text{\texttt{=}}}}}}}

\newcommand{\drawwireflow}[5]{%
    \draw[wireflow] (#4) to[#1] node[#2] {\scalebox{0.8}{\footnotesize\color{purple!60!blue!70} #3}} (#5);
}

\newcommand{\examplearrow}[1]{%
    \raisebox{0.25em}{\begin{tikzpicture}\draw[#1] (0,0) to (0.4,0);\end{tikzpicture}}%
}

%%% Local Variables:
%%% mode: plain-tex
%%% TeX-master: t
%%% End:
